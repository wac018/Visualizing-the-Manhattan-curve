\usepackage{amsmath, amsthm, amssymb, mathtools}
\usepackage{xcolor, soul}
\usepackage[english]{babel}
\usepackage{csquotes}
\usepackage[sortcites=true, sorting=nyt, backend=biber]{biblatex}
\usepackage{tikz, tikz-3dplot}
\usepackage{pgfplots}
\usepackage{listings}
\usepackage{comment}
%\usepackage{enumitem}
\usepackage{tcolorbox}
\usepackage{multicol}
\usepackage{todonotes}
%\usepackage{subcaption}
%\usepackage{pst-solides3d}

%\pgfplotsset{compat=1.18}
\usepgfplotslibrary{colormaps}
\pgfplotsset{
	compat=1.18,
	colormap={mycolormap}{color=(lightgray) color=(white) color=(lightgray)}
}

\usetikzlibrary{arrows.meta}
\usetikzlibrary{shapes.geometric}
\usetikzlibrary{calc}
\usetikzlibrary{decorations.pathreplacing,shapes.misc}
\usetikzlibrary{intersections}
\usepgfplotslibrary{fillbetween}

\graphicspath{{Images/}}

% --- Beamer Specific Stuff ---
\beamertemplatenavigationsymbolsempty

% Adds slide numbers to slides
\part{title}\setbeamertemplate{footline}[frame number]

%\usetheme{CambridgeUS}
%\setbeamertemplate{headline}{}

\AtBeginSection[]{
	\begin{frame}
		\vfill
		\centering
		\begin{beamercolorbox}[sep=8pt,center,shadow=true,rounded=true]{title}
			\usebeamerfont{title}{\insertsectionhead}\par%
		\end{beamercolorbox}
		\vfill
	\end{frame}
}


%\setbeameroption{hide notes} % Only slides
%\setbeameroption{show only notes} % Only notes
%\setbeameroption{show notes on second screen=right} % Both

% --- End of Beamer Specific Stuff ---

\renewcommand{\emph}{\textbf}


\definecolor{myorange}{HTML}{ff7f00} % Defines the color of the box around text for tcolorbox
\definecolor{myblue}{HTML}{46b8ff} % Defines a blue color for notes that I make
\definecolor{darkgreen}{HTML}{008000} % Defines a dark green for lines

\newcommand{\mytodo}[2][]{\todo[color=myblue,#1]{#2}} % Makes a custom todo box for notes that I make vs notes Giuseppe makes.

%https://osl.ugr.es/CTAN/macros/latex/contrib/tcolorbox/tcolorbox.pdf
\tcbuselibrary{breakable}
\tcbset{%any default parameters
	width=0.7\textwidth,
	halign=justify,
	center,
	breakable,
	colback=myorange    
}

% This controls how the code snippets are typset
\lstset{
	basicstyle=\ttfamily,
	columns=fullflexible,
	frame=single,
	breaklines=true,
	postbreak=\mbox{\textcolor{red}{$\hookrightarrow$}}\space
}

\lstset{language=Octave}

% Used to make the bar to make the notation for a restricted domain look nicer
\newcommand{\littletaller}{\mathchoice{\vphantom{\big|}}{}{}{}}

% Makes a command to automatically created the notation for a restricted domain
\newcommand\restrict[2]{
	{% we make the whole thing an ordinary symbol
		\left.\kern-\nulldelimiterspace % automatically resize the bar with \right
		#1 % the function
		\littletaller % pretend it's a little taller at normal size
		\right|_{#2} % this is the delimiter
	}
}


% Some nice math macros
\newcommand{\set}[1]{\left\{#1\right\}}
\newcommand*\diff{\mathop{}\!\mathrm{d}} % Changes the font of the differential d when writing dx or similar
\renewcommand{\st}{\colon} % renews the st command that is usually provided by soul to strike through
\newcommand{\N}{\mathbb{N}}
\newcommand{\R}{\mathbb{R}}
\newcommand{\Z}{\mathbb{Z}}
\newcommand{\Q}{\mathbb{Q}}
\newcommand{\RP}{\mathbb{RP}}
\newcommand{\Hyp}{\mathbb{H}}
\newcommand{\T}{\mathcal{T}}
\newcommand{\PSL}[2]{\mathsf{PSL}_{#1}\left(#2\right)}
\newcommand{\SL}[2]{\mathsf{SL}_{#1}\left(#2\right)}

\newcommand{\derof}[1]{\dot{#1}}
\newcommand{\der}{\mathsf{d}}
\newcommand{\crossratio}[1]{\operatorname{CR}\left[#1\right]}

\newcommand{\metricd}{\operatorname{d}}
\newcommand{\CR}[1]{\operatorname{CR}\left[#1\right]}
\newcommand{\bound}[1]{\partial #1}
\newcommand{\close}[1]{\overline{#1}}
\newcommand{\norm}[1]{\|#1\|}

\newcommand{\wrt}{\text{w.r.t. }}
\newcommand{\projof}[1]{\pi\left(#1\right)}
\newcommand{\proj}{\pi}
\newcommand{\abs}[1]{\left| #1 \right|}

% -- Tikz Pictures --

\tikzset{
	pics/torus/.style n args={3}{
		code = {
			\providecolor{pgffillcolor}{rgb}{1,1,1}
			\begin{scope}[
				yscale=cos(#3),
				outer torus/.style = {draw,line width/.expanded={\the\dimexpr2\pgflinewidth+#2*2},line join=round},
				inner torus/.style = {draw=pgffillcolor,line width={#2*2}}
				]
				\draw[outer torus] circle(#1);\draw[inner torus] circle(#1);
				\draw[outer torus] (180:#1) arc (180:360:#1);\draw[inner torus,line cap=round] (180:#1) arc (180:360:#1);
			\end{scope}
		}
	}
}

% -- End of Tikz Pictures --

% -- Tikz Styles --

\tikzset{%
	add/.style args={#1 and #2}{
		to path={%
			($(\tikztostart)!-#1!(\tikztotarget)$)--($(\tikztotarget)!-#2!(\tikztostart)$)%
			\tikztonodes},add/.default={.2 and .2}}
} 
% -- End of Tikz Styles --

% -- Tikz Functions --

\makeatletter
\newdimen\@XCoord
\newdimen\@YCoord
\newdimen\XCoord
\newdimen\YCoord
\newcommand*{\ExtractCoordinate}[1]{%
	\getscale{\@scalefactor}
	\path [transform canvas] (#1); \pgfgetlastxy{\@XCoord}{\@YCoord}
	\pgfmathsetlength{\XCoord}{\@XCoord/\@scalefactor}
	\pgfmathsetlength{\YCoord}{\@YCoord/\@scalefactor}
}
\newcommand*{\ExtractCoordinateOld}[1]{%
	\path [transform canvas] (#1); \pgfgetlastxy{\XCoord}{\YCoord}%
}%
%\let\ExtractCoordinate\ExtractCoordinateOld
\makeatother

\newdimen\McurveXcoord
\newdimen\McurveYcoord
% -- End of Tikz Functions --
\usepackage[english]{babel}
\usepackage{csquotes}
\usepackage[sortcites=true, sorting=nyt, backend=biber]{biblatex}
\usepackage[colorlinks=true,
pdfstartview=FitV, linkcolor=blue, citecolor=olive,
urlcolor=cyan]{hyperref}
\usepackage{xcolor, soul} %Package to insert links, citations, etc.
\usepackage[english]{babel}
\usepackage{csquotes}
\usepackage{amsmath, amsthm, amssymb, mathtools}
\usepackage{graphicx} % Required for inserting images
\usepackage{geometry}
\usepackage{multicol}
\usepackage{tabularx}
\usepackage{changepage}
\usepackage[T1]{fontenc}
\usepackage{tikz, pgfplots, tikz-3dplot}
\usepackage{listings} % Used for inserting code snippets
\usepackage{comment}
\usepackage{setspace}
\usepackage{todonotes}
\usepackage{tcolorbox} % used for drawing boxes around paragraphs so that notes are easier to identify

\usepackage{enumitem}
\usepackage{listings} % Used for including code snippets
\usepackage{subcaption}

\geometry{letterpaper, portrait, margin=2in}%1in}

\graphicspath{{Images/}}

\usetikzlibrary{external}
\tikzexternalize[prefix=tikz/]

% This makes todonotes work with tikz external
\makeatletter
\renewcommand{\todo}[2][]{\tikzexternaldisable\@todo[#1]{#2}\tikzexternalenable}
\makeatother



\definecolor{myorange}{HTML}{ff7f00} % Defines the color of the box around text for tcolorbox
\definecolor{myblue}{HTML}{46b8ff} % Defines a blue color for notes that I make
\definecolor{darkgreen}{HTML}{008000} % Defines a dark green for lines

\newcommand{\mytodo}[2][]{\todo[color=myblue, #1]{#2}} % Makes a custom todo box for notes that I make vs notes Giuseppe makes.

%https://osl.ugr.es/CTAN/macros/latex/contrib/tcolorbox/tcolorbox.pdf
\tcbuselibrary{breakable}
\tcbset{%any default parameters
	width=0.7\textwidth,
	halign=justify,
	center,
	breakable,
	colback=myorange    
}

% This controls how the code snippets are typset
\lstset{
	basicstyle=\ttfamily,
	columns=fullflexible,
	frame=single,
	breaklines=true,
	postbreak=\mbox{\textcolor{red}{$\hookrightarrow$}}\space
}

\lstset{language=Octave}


\newtheorem{theorem}{Theorem}[section]
\newtheorem{lemma}[theorem]{Lemma}

\theoremstyle{definition}
\newtheorem{definition}{Definition}[section]
\newtheorem*{example*}{Example}
\newtheorem{example}{Example}[section]


\theoremstyle{remark}
\newtheorem{remark}{Remark}

\newlist{eqlist}{enumerate}{2}
\setlist[eqlist]{label=(\roman*), before=\raggedright}

% Nice way to make notes for myself
\newcommand{\William}[1]{\textcolor{red}{[William] #1}}

% Some nice math macros
\newcommand{\set}[1]{\left\{#1\right\}}
\newcommand*\diff{\mathop{}\!\mathrm{d}} % Changes the font of the differential d when writing dx or similar
\renewcommand{\st}{\colon} % renews the st command that is usually provided by soul to strike through
\newcommand{\N}{\mathbb{N}}
\newcommand{\R}{\mathbb{R}}
\newcommand{\Z}{\mathbb{Z}}
\newcommand{\Q}{\mathbb{Q}}
\newcommand{\RP}{\mathbb{RP}}
\newcommand{\Hyp}{\mathbb{H}}
\newcommand{\T}{\mathcal{T}}
\newcommand{\PSL}[2]{\mathsf{PSL}_{#1}\left(#2\right)}
\newcommand{\SL}[2]{\mathsf{SL}_{#1}\left(#2\right)}

\newcommand{\derof}[1]{\dot{#1}}
\newcommand{\der}{\mathsf{d}}
\newcommand{\crossratio}[1]{\operatorname{CR}\left[#1\right]}

\newcommand{\metricd}{\operatorname{d}}
\newcommand{\CR}[1]{\operatorname{CR}\left[#1\right]}
\newcommand{\bound}[1]{\partial #1}
\newcommand{\hvec}[2]{\begin{bmatrix} #1 \\ #2 \end{bmatrix}}

% Makes a todo list environment to make nice todo lists for things that would need to be shown
\newlist{todolist}{itemize}{2}
\setlist[todolist]{label=$\square$}

\newcommand{\inquotes}[1]{``#1''}

\newcommand{\close}[1]{\overline{#1}}

\newcommand{\wrt}{\text{w.r.t. }}
\newcommand{\projof}[1]{\pi\left(#1\right)}
\newcommand{\proj}{\pi}

% Used to make the bar to make the notation for a restricted domain look nicer
\newcommand{\littletaller}{\mathchoice{\vphantom{\big|}}{}{}{}}

% Makes a command to automatically created the notation for a restricted domain
\newcommand\restrict[2]{
	{% we make the whole thing an ordinary symbol
		\left.\kern-\nulldelimiterspace % automatically resize the bar with \right
		#1 % the function
		\littletaller % pretend it's a little taller at normal size
		\right|_{#2} % this is the delimiter
	}
}

%\frenchspacing
%\doublespacing

\newenvironment{summary}{
	\hfill \break 
	\begin{center}
		\textbf{Summary:}
	\end{center}
	\begin{adjustwidth}{0.5in}{0.5in}
		
	
}{
\end{adjustwidth}
\bigskip
}

\usetikzlibrary{arrows.meta}
\usetikzlibrary{shapes.geometric}
\usetikzlibrary{calc}
\usetikzlibrary{decorations.pathreplacing,shapes.misc}
\usetikzlibrary{intersections}
\usepgfplotslibrary{fillbetween}
\usepgfplotslibrary{colormaps}

\pgfplotsset{
	compat=1.18,
	colormap={mycolormap}{color=(lightgray) color=(white) color=(lightgray)}
}

\tikzset{
	convexset/.style = {line width = 0.75 pt, fill = white},
	ext/.style = {circle, inner sep=0pt, minimum size=2pt, fill=black},
	segment/.style = {line width = 0.75 pt}
}


% -- Tikz Pictures --

\tikzset{
	pics/torus/.style n args={3}{
		code = {
			\providecolor{pgffillcolor}{rgb}{1,1,1}
			\begin{scope}[
				yscale=cos(#3),
				outer torus/.style = {draw,line width/.expanded={\the\dimexpr2\pgflinewidth+#2*2},line join=round},
				inner torus/.style = {draw=pgffillcolor,line width={#2*2}}
				]
				\draw[outer torus] circle(#1);\draw[inner torus] circle(#1);
				\draw[outer torus] (180:#1) arc (180:360:#1);\draw[inner torus,line cap=round] (180:#1) arc (180:360:#1);
			\end{scope}
		}
	}
}

% --- End of Tikz Pictures ---


% -- Tikz Functions --

\tikzset{%
	add/.style args={#1 and #2}{
		to path={%
			($(\tikztostart)!-#1!(\tikztotarget)$)--($(\tikztotarget)!-#2!(\tikztostart)$)%
			\tikztonodes},add/.default={.2 and .2}}
} 

% https://tex.stackexchange.com/questions/86897/recover-scaling-factor-in-tikz
\newcommand*\getscale[1]{%
	\begingroup
	\pgfgettransformentries{\scaleA}{\scaleB}{\scaleC}{\scaleD}{\whatevs}{\whatevs}%
	\pgfmathsetmacro{#1}{sqrt(abs(\scaleA*\scaleD-\scaleB*\scaleC))}%
	\expandafter
	\endgroup
	\expandafter\def\expandafter#1\expandafter{#1}%
}


% -- End of Tikz Functions --